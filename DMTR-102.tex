\documentclass[DM,lsstdraft,STR,toc]{lsstdoc}
\usepackage{geometry}
\usepackage{longtable,booktabs}
\usepackage{enumitem}
\usepackage{arydshln}

\input meta.tex

\providecommand{\tightlist}{
  \setlength{\itemsep}{0pt}\setlength{\parskip}{0pt}}

\begin{document}

\def\milestoneName{Small Scale CCOB Data Access}
\def\milestoneId{503-8b}
\def\product{Archiving}

\setDocCompact{true}

\title[\milestoneId{}~Test Report]{\milestoneId{} (\milestoneName{})~Test Plan and Report}
\setDocRef{\lsstDocType-\lsstDocNum}
\setDocDate{\vcsdate}
\setDocUpstreamLocation{\url{https://github.com/lsst/lsst-texmf/examples}}
\author{ Michelle Butler }

\input history_and_info.tex


\setDocAbstract{
This is the test plan and report for \milestoneId{} (\milestoneName{}), an LSST level 2 milestone pertaining to the \product{} component provided by the Data Management Subsystem.
}


\maketitle

\section{Introduction}
\label{sect:intro}


\subsection{Objectives}
\label{sect:objectives}

This test activity will demonstrate the successful execution of the
major components required to transfer raw image files from the DAQ to
the DBB and subsequently visualize them via the LSST Science
Platform(LSP). This is a small scale proof-of-concept test designed to
demonstrate that the key components can interoperate
successfully.\\[2\baselineskip]It will specifically demonstrate that:

\begin{itemize}
\tightlist
\item
  Data can be transferred into the lsst data facility (LDF)
\item
  The raw data images can be ingested by the DBB along with a subset of
  image and provenance metadata
\item
  The data files can then be retrieved/viewed by the LSP
\end{itemize}

\subsection{Scope}\label{scope}

The overall test plan for the LSST Data Management system is described
in LDM-503\\
This document specifically refers to the milestone LDM-503-8b, which
tests the raw image archiving for a CCOB device



\subsection{System Overview}
\label{sect:systemoverview}

The raw image archiving system that is part of the LSST Data Management
system is responsible for the data being stored into the archive
environment at the LDF from various sources. ~It will depend on what DAQ
system (spectragraph, comcam, lsstcam, CCOB\ldots{} others) as to where
the data is located originally, and then how it will be sent to NCSA
(LDF). The tests begin with data somewhere that needs to be stored at
the archive environment at LDF and ingested into the DBB. The LDM-503-4
milestone is limited to the raw image archive from a CCOB device,
specifically the DAQ, not all types of devices are included and is
limited in scope.\\[2\baselineskip]

\subsection{Applicable Documents}\label{applicable-documents}

\citeds{LDM-294} Data Management Organization and Management\\
\citeds{LDM-503} DM Test Plan\\
\citeds{LDM-148} Data Management System Design


\subsection{Document Overview}
\label{sect:docoverview}

This document was generated from Jira, obtaining the relevant information from the 
\href{https://jira.lsstcorp.org/secure/Tests.jspa#/testPlan/LVV-P10}{LVV-P10}
~Jira Test Plan and related Test Cycles (
  \href{https://jira.lsstcorp.org/secure/Tests.jspa#/testCycle/LVV-C8}{LVV-C8}
).

%The following general sections are completed before the start of the test activity.

Section \ref{sect:intro} provides an overview of the test campaign, the system under test (\product{}), the applicable documentation, and explains how this document is organized.
Section \ref{sect:configuration}  describes the configuration used for this test.
Section \ref{sect:personnel} lists all the individuals involved and describes their roles.
%Section \ref{sect:plannedtestactivities} provides the list of planned test cycles and test cases, including all relevant information that fully describes the test campaign.

Section \ref{sect:overview} provides a summary of the test results, including an overview in Table \ref{table:summary}, an overall assessment statement and suggestions for possible improvements.
Section \ref{sect:detailedtestresults} provides detailed results for each step in each test case.

The current status of test plan LVV-P10 in Jira is Draft.

\subsection{References}
\label{sect:references}
\renewcommand{\refname}{}
\bibliography{lsst,refs,books,refs_ads}
\section{Test Configuration}
\label{sect:configuration}

\subsection{Data Collection}

  Observing is not required for this test campaign.

\subsection{Verification Environment}
\label{sect:hwconf}
  LDF center at NCSA. The archive service or data back bone.


  \subsection{Entry Criteria}
  Data that needs to be stored from a CCOB or DAQ


  \subsection{Exit Criteria}
  Data are correctly transferred to the LDF and ingested into the DBB, and
can be subsequently viewed on the LSP.



\section{Personnel}
\label{sect:personnel}

The following personnel are involved in this test activity:

\begin{itemize}
\item Test Plan (LVV-P10) owner: Michelle Butler
\item Test Cycles:
\begin{itemize}
  \item LVV-C8 owner: 
    Undefined
  \begin{itemize}
    \item Test case LVV-T284 tester: Michelle Butler
  \end{itemize}
\end{itemize}
\item Additional Test Personnel involved: None
\end{itemize}

\newpage

\section{Overview of the Test Results}
\label{sect:overview}

\subsection{Summary}
\label{sect:summarytable}

\begin{longtable} {p{0.2\textwidth}p{0.2\textwidth}p{0.6\textwidth}}
\toprule
  \multicolumn{3}{c}{ Test Cycle {\bf LVV-C8: 503-8b raw image archiving for CCOB cycle }} \\\hline
  {\bf \footnotesize test case} & {\bf \footnotesize status} & {\bf \footnotesize comment} \\\toprule
    \href{https://jira.lsstcorp.org/secure/Tests.jspa#/testCase/LVV-T284}{LVV-T284} 
    & Not Executed & 

    \\\hline

\caption{Test Results Summary}
\label{table:summary}
\end{longtable}

\subsection{Overall Assessment}
\label{sect:overallassessment}

Not yet available.

\subsection{Recommended Improvements}
\label{sect:recommendations}

Not yet available.

\newpage
\section{Detailed Test Results}
\label{sect:detailedtestresults}


  \subsection{Test Cycle LVV-C8 }

Open test cycle {\it \href{https://jira.lsstcorp.org/secure/Tests.jspa#/testrun/LVV-C8}{503-8b raw image archiving for CCOB cycle}} in Jira.

  503-8b raw image archiving for CCOB cycle\\
  Status: Not Executed

  Test cycle for the raw image archiving of data from a CCOB device. ~This
data needs to be stored at the LDF and ingested so that it can be looked
at by scientists. ~ The data needs to be transferred to the LDF and then
ingested into the archive service so that it can be retrieved through a
butler to be displayed for further use. ~ This should be tested with a
human involved at first for moving the data and making sure all the
pieces work. ~ It should also be tested for when things are working more
as a service, and without a human involved for an automatic data archive
process.\\[2\baselineskip]


  \subsubsection{Software Version/Baseline}
    The DBB version used in this campaign is 1.0, and can be found in the
github repository when it has been fully developed. ~ The current DBB is
a script for scraping data from the header file and getting the file
name and the metadata required along with provenence. ~ At the time of
this writing, the DBB use cases and requirements are just being written.
~ ~The database component is Oracle and the ingest file systems and
output file systems for DBB are in GPFS at the LDF. ~


  \subsubsection{Configuration}
    The configuration for this test is the data is located in a file system
somewhere. ~It is transferred to LSST GPFS file system at LDF. Files
that are needed in the DBB are moved to the DBB ingest file system. The
data backbone gathers the necessary information from the headers and
file name. It inserts rows into the DBB consolidated DB environment and
moves the file to the DBB file systems that are used by the LSP to once
again find the raw image file to be inspected for further analysis.
~\\[2\baselineskip]All files from the CCOB are to be transferred to
NCSA. ~ A directory of where the CCOB files are stored is required so
that the directory can be monitored for when new files are written, they
can be transferred. ~ ~At the first iteration, a human will point out
where the files are, and begin the execution of the ``rsync'' process to
bring all the files to the LDF. ~ ~This is so that the files can be
inspected by scientists to ensure that all the files and data have been
transferred. ~ ~ This test will be run with humans involved, and then
will be run as a integrated test of the whole infrastructure. ~


  \subsubsection{Test Cases in LVV-C8 Test Cycle}


    \paragraph{Test Case LVV-T284 }\mbox{}\\

Open  \href{https://jira.lsstcorp.org/secure/Tests.jspa#/testCase/LVV-T284}{\textit{ LVV-T284 } }
test case in Jira.

    This test will check:

\begin{itemize}
\tightlist
\item
  The successful integration of the DAQ archiver components with the
  CCOB
\item
  That the file can then be ingested into the DBB and be retrieved for
  further analysis
\end{itemize}


    {\bf Preconditions}:\\
    None.


    Execution status: {\bf Not Executed }

    Final comment:\\



    Detailed step results:

    \begin{longtable}{p{1cm}p{2cm}p{13cm}}
    \hline
    {Step} & \multicolumn{2}{c}{Description, Results and Status}\\ \hline
      1 & Description &

      \begin{minipage}[t]{13cm}{\footnotesize
      CCOB device directs a human to where a raw file is wanted to be stored
in the DBB

      \vspace{\dp0}
      } \end{minipage} \\
      \\ \cdashline{2-3}

      & Expected Result & 

      \begin{minipage}[t]{13cm}{\footnotesize
      A file with a unique file name is in a file system somewhere, and the
data is then transferred to NCSA.~ ~

      \vspace{\dp0}
      } \end{minipage} \\
      \\ \cdashline{2-3}

      & \begin{minipage}[t]{2cm}{Actual\\ Result}\end{minipage}   & 
      \begin{minipage}[t]{13cm}{\footnotesize
      
      \vspace{\dp0}
      } \end{minipage} \\
      \\ \cdashline{2-3}

      & Status          & Not Executed \\ \hline

      2 & Description &

      \begin{minipage}[t]{13cm}{\footnotesize
      Move the data from the transferred directory into the DBB foreign file
ingest file system. ~

      \vspace{\dp0}
      } \end{minipage} \\
      \\ \cdashline{2-3}

      & Expected Result & 

      \begin{minipage}[t]{13cm}{\footnotesize
      A command is executed by a human with a file name and path to the file
wanted to be stored in the DBB.~ The file is transferred to NCSA's DBB
ingest area.~ ~~

      \vspace{\dp0}
      } \end{minipage} \\
      \\ \cdashline{2-3}

      & \begin{minipage}[t]{2cm}{Actual\\ Result}\end{minipage}   & 
      \begin{minipage}[t]{13cm}{\footnotesize
      
      \vspace{\dp0}
      } \end{minipage} \\
      \\ \cdashline{2-3}

      & Status          & Not Executed \\ \hline

      3 & Description &

      \begin{minipage}[t]{13cm}{\footnotesize
      The DBB is notified of a new file being in the ingest area, and the DBB
ingest is run manually to ingest the CCOB file.~ ~

      \vspace{\dp0}
      } \end{minipage} \\
      \\ \cdashline{2-3}

      & Expected Result & 

      \begin{minipage}[t]{13cm}{\footnotesize
      The DBB puts the resulting file into the DBB file systems depending on
what type of file it is. ~The DB is updated with metadata and providence
of the file to be kept. ~ The resulting file system is queryable by the
LSP to find the CCOB raw image.~~

      \vspace{\dp0}
      } \end{minipage} \\
      \\ \cdashline{2-3}

      & \begin{minipage}[t]{2cm}{Actual\\ Result}\end{minipage}   & 
      \begin{minipage}[t]{13cm}{\footnotesize
      
      \vspace{\dp0}
      } \end{minipage} \\
      \\ \cdashline{2-3}

      & Status          & Not Executed \\ \hline

      4 & Description &

      \begin{minipage}[t]{13cm}{\footnotesize
      The LSP can review and use the CCOB raw data file that was stored
originally somewhere else such as slac

      \vspace{\dp0}
      } \end{minipage} \\
      \\ \cdashline{2-3}

      & Expected Result & 

      \begin{minipage}[t]{13cm}{\footnotesize
      LSP has the ability to find the file and view/use it.
~\\[2\baselineskip]

      \vspace{\dp0}
      } \end{minipage} \\
      \\ \cdashline{2-3}

      & \begin{minipage}[t]{2cm}{Actual\\ Result}\end{minipage}   & 
      \begin{minipage}[t]{13cm}{\footnotesize
      
      \vspace{\dp0}
      } \end{minipage} \\
      \\ \cdashline{2-3}

      & Status          & Not Executed \\ \hline

      5 & Description &

      \begin{minipage}[t]{13cm}{\footnotesize
      Have data inspected by scientist for managing that all data was
transferred.~ ~

      \vspace{\dp0}
      } \end{minipage} \\
      \\ \cdashline{2-3}

      & Expected Result & 

      \begin{minipage}[t]{13cm}{\footnotesize
      a specific Okay to move forward; or something is
broke.\\[2\baselineskip]

      \vspace{\dp0}
      } \end{minipage} \\
      \\ \cdashline{2-3}

      & \begin{minipage}[t]{2cm}{Actual\\ Result}\end{minipage}   & 
      \begin{minipage}[t]{13cm}{\footnotesize
      
      \vspace{\dp0}
      } \end{minipage} \\
      \\ \cdashline{2-3}

      & Status          & Not Executed \\ \hline

    \end{longtable}


\input{appendix.tex}
\end{document}
