\documentclass[DM,lsstdraft,STR,toc]{lsstdoc}
\usepackage{geometry}
\usepackage{longtable,booktabs}

\input meta.tex

\begin{document}

\def\milestoneName{Raw Image Archiving}
\def\milestoneId{LDM-503-8b}
\def\product{LSST Raw Image Archiving System}

\setDocCompact{true}

\title[\milestoneId{}~Test Report]{\milestoneId{} (\milestoneName{})~Test Plan and Report}
\setDocRef{\lsstDocType-\lsstDocNum}
\setDocDate{\vcsdate}
\setDocUpstreamLocation{\url{https://github.com/lsst/lsst-texmf/examples}}
\author{% `git log --pretty=%an | sort --key=2 | uniq` ?
  Michelle Butler
}

% Most recent last
\setDocChangeRecord{
\addtohist{}{2018-10-31}{Test plan Update}{Butler, Comoretto}
\addtohist{}{2018-10-10}{Initial version.}{Butler, Comoretto}
}

\setDocCurator{Michelle Butler}
\setDocUpstreamLocation{\url{https://github.com/lsst-dm/\lsstDocType-\lsstDocNum}}
\setDocUpstreamVersion{\vcsrevision}


\setDocAbstract{
This is the test plan and report for \milestoneId{} (\milestoneName{}), an LSST DM level 1 milestone pertaining to the \product{}.
}

\maketitle

\section{Introduction}
\label{sect:intro}


\subsection{Objectives}
\label{sect:objectives}

Testing for the raw image from the DAQ to being written to the archive service at the LDF
This test activity demonstrates the successful execution of the major components of the storing of raw image files from the DAQ to the DBB and then visualized through the LSST science platform (LSP).   This is at a very small scale and it to test the components of a proof of concept that they all work together.  

It will demonstrate
\begin{itemize}
\item Data can be transferred into the lsst data facility (LDF).   
\item The raw data images can be ingested by the DBB along with some metadata about the image and providence.  
\item The data files  can then be retrieved by the LSP
\end{itemize}

\subsection{Scope}
\label{sect:scope}

The overall test plan for the LSST Data Management system is described in LDM-503.
This document specifically refers to the milestone LDM-503-8b, which tests the raw image archiving for a CCOB device.  

\subsection{System Overview}
\label{sect:systemoverview}

The raw image archiving system that is part of the LSST Data Management system is responsible for the data being stored into the archive environment at the LDF from various sources.  It will depend on what DAQ system (spectragraph, comcam, lsstcam, CCOB... others) as to where the data is located originally, and then how it will be sent to NCSA (LDF) .    The tests begin with data somewhere that needs to be stored at the archive enviornment at LDF and ingested into the DBB.   The LDM-503-5 milestone is limited to the raw image archive from a CCOB device, not all types of devices and is limited in scope.  

\subsection{Applicable Documents}
\label{sect:appdocs}
\addtocounter{table}{-1}

\begin{tabular}[htb]{l l}
\citeds{LDM-294} & LSST DM Project Management Plan\\
\citeds{LDM-503} & DM Test Plan\\
\citeds{LDM-538} & DM Raw Image Archiving Service Test Specification \\
\end{tabular}

The information contained in this document is autogenerated from Jira.
The following Jira test plan and test cycle(s) are used to collect
such information:

\begin{itemize}
\item Test Plan: \href{https://jira.lsstcorp.org/secure/Tests.jspa\#/testPlan/LVV-P10}{LVV-P10} 503-8b raw image archiving for data from CCOB device 
\item Test Cycle: \href{https://jira.lsstcorp.org/secure/Tests.jspa\#/testCycle/LVV-C8}{LVV-C8} 503-8b raw image archiving for CCOB cycle
\end{itemize}

 
\subsection{References}
\label{sect:references}

\renewcommand{\refname}{}
\bibliography{lsst,refs,books,refs_ads}

\subsection{Document Overview}
\label{sect:docoverview}

The following planning sections are completed before the start of the test activity.
Section \ref{sect:configuration} of this document provides details of the \product{} baseline used for this test, including relevant hardware and software configurations.
Section \ref{sect:personnel} lists the individuals involved in performing the tests.
Section \ref{sect:plannedtestactivities} provides a descriptive list of planned test cases.
Once the above sections are completed, this document can be reviewd in order to ensure that the test activity can start.

Section \ref{sect:testresults} is filled after the test activity is completed. 
Its includes  an overview of the resulta in \ref{sect:overview}
while \ref{sect:detailedtestresults} provides more detailed results from each individual test case.

\section{Test Configuration}
\label{sect:configuration}

The configuration for this test is the data is located in a file system somewhere.  It is transferred to NCSA into a the LSST GPFS file system.   Files that are needed in the DBB are moved to the DBB ingest file system.   the data backbone gathers the info from the headers and file name.  it inserts rows into the DBB consolidated DB environment and moves the file to the DBB file systems that are used by the LSP to once again find the raw image file to be viewed/displayed for further analysis.  

\subsection{Hardware}
\label{sect:hwconf}

LDF center at NCSA.   The archive service or data back bone.

\subsection{Software}
\label{sect:swconf}

The DBB version is 1.0, and can be found in the github repository when it has been fully developed.   The current DBB is a script for scraping data from the header file and getting the file name and the metadata required along with providence.   At the time of this writing, the DBB use cases and requirements are just being written.    The database component is Oracle and the ingest file systems and output file systems for DBB are in GPFS at the LDF.  

\subsection{Entry Criteria}

Data that needs to be stored from a CCOB or DAQ.

\subsection{Exit Criteria}

Data is transferred in proper form and can be viewed on LSP after being transferred to LDF and ingested into DBB.  


\section{Personnel}
\label{sect:personnel}

Following personnel is involved in the test activity:

\begin{itemize}
\item Test Plan (LVV-P10) Owner: Michelle Butler
\item Test Cycle (LVV-C8) Owner: Unassigned
\item LVV-T284 Tester: Michelle Butler
\item Additional Test Personel involved: None 
\end{itemize}

\newpage


\section{Planned Test Activities}
\label{sect:plannedtestactivities}

\subsection{Test Cycle LVV-C8}

Test cycle for the raw image archiving of data from a CCOB device.  This data needs to be stored at the LDF and ingested so that it can be looked at by scientists.   The data needs to be transferred to the LDF and then ingested into the archive service so that it can be retrieved through a butler to be displayed for further use.   This should be tested with a human involved at first for moving the data and making sure all the pieces work.   It should also be tested for when things are working more as a service, and without a human involved for an automatic data archive process.   

\subsubsection{LVV-T284: Writing data from CCOB to the DBB for further data processing}

This test will check:
The successful integration of the Pathfinder components with the CCOB;
That the file can then be ingested into the DBB and be retrieved for further analysis;

    \begin{longtable}[]{p{1.3cm}p{2cm}p{13cm}}
    %\toprule
    Step & \multicolumn{2}{@{}l}{Description} \\ \toprule
    \endhead

            \multirow{1}{*}{ 1 } &  &
            \begin{minipage}[t]{13cm}{\footnotesize
            CCOB device directs a human to where a file is wanted to be stored in
the DBB

            \vspace{\dp0}
            } \end{minipage} \\ \cline{2-3}

            \multirow{1}{*}{ 2 } &  &
            \begin{minipage}[t]{13cm}{\footnotesize
            Move the data from the transferred directory into the DBB foreign file ingest file system.

            \vspace{\dp0}
            } \end{minipage} \\ \cline{2-3}

            \multirow{1}{*}{ 3 } &  &
            \begin{minipage}[t]{13cm}{\footnotesize
            The DBB is notified of a new file being in the ingest area, and the DBB ingest is run manually to ingest the CCOB file.

            \vspace{\dp0}
            } \end{minipage} \\ \cline{2-3}

            \multirow{1}{*}{ 4 } &  &
            \begin{minipage}[t]{13cm}{\footnotesize
            The LSP can review and use the CCOB raw data file that was stored originally somewhere else such as slac

            \vspace{\dp0}
            } \end{minipage} \\ \cline{2-3}

        \\ \midrule
    \end{longtable}



\newpage

\section{Test Results}
\label{sect:testresults}

\subsection{Overview of the Test Results}
\label{sect:overview}

\subsubsection{Summary Table}
\label{sect:summarytable}


\begin{longtable} {|p{0.2\textwidth}|p{0.2\textwidth}|p{0.6\textwidth}|}
\hline
{\bf TEST CASE ID} & {\bf PASS/FAIL} & {\bf COMMENTS} \\\hline
LVV-T284 & Not Run & \\\hline
\end{longtable}

\subsubsection{Overall Assessment}
\label{sect:overallassessment}


\subsubsection{Recommended Improvements}
\label{sect:recommendations}

\subsection{Detailed Test Results}
\label{sect:detailedtestresults}

\end{document}
